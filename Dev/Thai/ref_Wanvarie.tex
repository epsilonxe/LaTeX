\documentclass[a4paper,11pt]{article}

% Set the locale for linebreak to Thai
\XeTeXlinebreaklocale "th"
% In English, when TeX tries to justify text,
% it will add some spaces between words
% For Thai, we "must not" add any space between words
% i.e. put "zero" space between words
\XeTeXlinebreakskip = 0pt plus 0pt
% For a bit better(?) justified output
\sloppy

% For any unicode characters, require XeTeX/XeLaTeX
\usepackage{fontspec}
\defaultfontfeatures{Mapping=tex-text}

% Set main fonts
% For Thai, I recommend to scale the size to the uppercase size of latin alphabet
%\setmainfont[Scale=MatchUppercase,Mapping=tex-text]{TH Sarabun New}
\setmainfont{TeX Gyre Termes}				% Free Times

% Sans font
\setsansfont{TeX Gyre Heros}				% Free Helvetica

% Monospace font
\setmonofont{TeX Gyre Cursor}				% Free Courier

% Because latin font in Sarabun is Sans Serif, we prefer to use Serif font
\newfontfamily{\thaifont}[Scale=MatchUppercase,Mapping=tex-text]{TH Sarabun New}

% Set environment for Thai fonts
\newenvironment{thailang}
{\thaifont}
{}

% For automatic switching between languages
\usepackage[Latin,Thai]{ucharclasses}

% When using Thai characters use thailang environment
\setTransitionTo{Thai}{\begin{thailang}}
% For other characters, switch back to the original environment
\setTransitionFrom{Thai}{\end{thailang}}

% Single spacing is too tight for Thai
\usepackage{setspace}
\onehalfspacing

% For thaialph numbering \thAlph
\usepackage{polyglossia}
% Set the normal language to English
% i.e. numbering, latin characters will use English font
\setdefaultlanguage{english}
% When using Thai characters, the font will be automatically changed to Thai font
\setotherlanguage{thai}

\AtBeginDocument\captionsthai               % Force the caption to Thai

% More references in gloss-thai of polyglossia package
% http://texdoc.net/texmf-dist/tex/latex/polyglossia/gloss-thai.ldf
% \thaiAlpha works well, the sequence is ก ข "ฃ" ค "ฅ" ฆ ง จ ...
% Normally, we need ก ข ค ง จ which is defined in \thaialph
% I'm not sure why it doesn't work. So, I just re-define it.
\def\thaialph#1{\expandafter\thalph\csname c@#1\endcsname}
\def\thalph#1{%
    \ifcase#1\or ก\or ข\or ค\or ง\or จ\or ฉ\or ช\or ซ\or
    ฌ\or ญ\or ฎ\or ฏ\or ฐ\or ฑ\or ฒ\or ณ\or ด\or ต\or ถ\or ท\or ธ\or น\or
    บ\or ป\or ผ\or ฝ\or พ\or ฟ\or ภ\or ม\or ย\or ร\or ฤ\or ล\or ฦ\or ว\or
    ศ\or ษ\or ส\or ห\or ฬ\or อ\else ฮ\else\xpg@ill@value{#1}{thalph}\fi}

% Again, re-define the sequence of Thai number.
\def\thainum#1{\expandafter\thainumber\csname c@#1\endcsname}
\def\thainumber#1{%
    \thaidigits{\number#1}%
}
\def\thaidigits#1{\expandafter\thdigits #1@ }
\def\thdigits#1{%
    \ifx @#1% then terminate
    \else
    \ifx0#1๐\else\ifx1#1๑\else\ifx2#1๒\else\ifx3#1๓\else\ifx4#1๔\else\ifx5#1๕\else\ifx6#1๖\else\ifx7#1๗\else\ifx8#1๘\else\ifx9#1๙\else#1\fi\fi\fi\fi\fi\fi\fi\fi\fi\fi
    \expandafter\thdigits
    \fi
}

%%%%%%%%%%%%%%%%%%%%%%%%%%%%%%%%%%%%%%%%%%%%%%%%%%%%%%%%%%%%%%%%%%%%%%%%%%
\usepackage[top=2cm, bottom=2cm, left=2cm, right=2cm]{geometry}

\usepackage{listings}
\lstset{
    language=TeX,
    basicstyle=\ttfamily,
    numbers=left,
    numberstyle=\small,
    breaklines=true,
    xleftmargin=.1\linewidth,
    frame=single,
    columns=fullflexible,
    captionpos=b,
    showstringspaces=false,
}


\renewcommand{\thesection}{\thainum{section}}
\title{คำแนะนำการตั้งค่า \LaTeX~สำหรับใช้ภาษาไทย}
\author{ฑิตยา หวานวารี}
\begin{document}
    \maketitle

    เราจะเรียงพิมพ์ด้วย Xe\TeX~ หรือ Xe\LaTeX~ เนื่องจากทั้งสองโปรแกรมนี้รองรับ Unicode ทำให้สามารถใช้ภาษาไทยปนกับภาษาอังกฤษได้สะดวก โปรแกรมทั้งสองนี้จะเรียงพิมพ์แล้วได้ผลลัพธ์ออกมาเป็น pdf เหมือนกับ pdf\TeX~ และ pdf\LaTeX~ แต่ยังคงสามารถแทรกรูปภาพประกอบเป็น pdf eps ได้ตามปกติ

\section{Package ที่จำเป็น}
\renewcommand{\theenumi}{\thaialph{enumi}}
\begin{enumerate}
    \item fontspec
    \item polyglossia
    \item ucharclass
\end{enumerate}

\section{การตั้งค่า}
ใส่คำสั่งต่อไปนี้ลงใน preamble
\begin{lstlisting}[language=]
\XeTeXlinebreaklocale "th"
\XeTeXlinebreakskip = 0pt plus 0pt

\usepackage{fontspec}
\defaultfontfeatures{Mapping=tex-text}
\setmainfont{TeX Gyre Termes}				% Free Times
\setsansfont{TeX Gyre Heros}				% Free Helvetica
\setmonofont{TeX Gyre Cursor}				% Free Courier

\newfontfamily{\thaifont}[Scale=MatchUppercase,Mapping=tex-text]{TH Sarabun New}
\newenvironment{thailang}
{\thaifont}
{}

\usepackage[Latin,Thai]{ucharclasses}
\setTransitionTo{Thai}{\begin{thailang}}
\setTransitionFrom{Thai}{\end{thailang}}

\usepackage{setspace}
\onehalfspacing

\usepackage{polyglossia}
\setdefaultlanguage{english}
\setotherlanguage{thai}

\AtBeginDocument\captionsthai               % Force the caption to Thai
\end{lstlisting}

บรรทัดที่ 1 และ 2 สำหรับการตัดคำภาษาไทย โดยกำหนด locale ให้ Xe\TeX~ รู้ว่าเป็นภาษาไทย
โดยปกติแล้ว \TeX~ จะจัดหน้าแบบชิดขอบซ้ายขวา (Justified) ซึ่งในภาษาอังกฤษนั้นจะปรับด้วยการเพิ่มช่องว่างระหว่างคำให้กว้างขึ้น แต่ในภาษาไทยเราไม่มีการเว้นวรรคระหว่างคำ จึงต้องตั้งค่านี้ให้เป็น 0 เพื่อให้ \TeX~ ไปเพิ่มช่องว่างตามเคาะวรรคที่เราเคาะแทน เนื่องจากการตัดคำภาษาไทยยังไม่ดีนัก การพยายามจัดหน้าอาจจะต้องหาทางเคาะวรรคตัดข้อความยาวๆ เองบ้าง เพื่อช่วยในการจัดหน้า หรือใช้ macro \lstinline|\sloppy| เพื่อช่วยในการถ่างช่องไฟระหว่างข้อความให้กว้างขึ้น\footnote{เอกสารฉบับนี้ใช้ macro \lstinline|\\sloppy| ช่วยจัดหน้า}

บรรทัดที่ 4 เป็นการเรียกใช้ package fontspec เพื่อให้เลือกใช้ font ภาษาไทยได้
บรรทัดที่ 5 เป็นการกำหนด option ของ font ให้ map อักขระต่างๆ แบบ \TeX~ เช่นการใส่ backquote/singleqoute ``รอบข้อความ'' บรรทัดที่ 6-8 เป็นการกำหนด font หลักของเอกสาร โดยแบ่งออกเป็น 3 ประเภท คือ
\renewcommand{\theenumi}{\arabic{enumi}}
\begin{enumerate}
    \item Serif กำหนดโดย \lstinline|\setmainfont| ซึ่งค่าเริ่มต้นโดยปกติจะเป็น Computer Modern แต่เพื่อให้ font เป็นไปในทางเดียวกับเอกสารที่พิมพ์ด้วย Word processor อื่นๆ ในที่นี้จึงกำหนดให้เป็น TeX Gyre Termes ซึ่งเป็น font Times ที่แจกฟรี (Times New Roman ใน Windows ไม่ฟรี และอาจไม่มีใน Linux)
    \item Sans Serif กำหนดโดย \lstinline|\setsansfont|
    \item Monospace เช่นพวก source code ต่างๆ กำหนดโดย \lstinline|\setmonofont|
\end{enumerate}

จริงๆ แล้วหากใช้ font เดียวกันทั้งเอกสาร สามารถตัดบรรทัดที่ 10-15 นี้ออกได้ แล้วกำหนด main font (บรรทัดที่ 6) ให้เป็น font สำหรับภาษาไทยก็เพียงพอ แต่ font ภาษาไทย โดยเฉพาะสารบัญนั้น ในภาษาอังกฤษเป็น font แบบไม่มีเชิง (Sans Serif) ซึ่งบางครั้งก็ไม่เหมาะกับการพิมพ์เอกสารแบบเป็นทางการ ในที่นี้จึงกำหนดให้ font ภาษาอังกฤษเป็น Times เพื่อให้เป็น font ที่มีเชิง (Serif) แล้วใช้ package ucharclasses ในการสลับ font อัตโนมัติแทน

บรรทัดที่ 10 เป็นการกำหนดชื่อ font อ้างอิง เพื่อให้สะดวกในการสลับภาษาอัตโนมัติ การสลับภาษาอัตโนมัตินี้ใช้ package ucharclasses โดยกำหนดชุดอักขระที่จะใช้เป็น option ในที่นี้คือ Latin และ Thai ดังในบรรทัดที่ 15 ส่วนบรรทัดที่ 16 เป็นการบอกว่าเมื่อพบอักขระภาษาไทย ให้ทำคำสั่งในวงเล็บปีกกา ซึ่งคือการเริ่มต้นใช้สภาพแวดล้อม \lstinline|thailang| และบรรทัดที่ 17 คือเมื่อพบอักขระที่ไม่ใช่ภาษาไทย (นอกขอบเขต unicode ของภาษาไทย) ให้สิ้นสุดการทำงานในสภาพแวดล้อม \lstinline|thailang| โดยสภาพแวดล้อม \lstinline|thailang| นั้นประกาศไว้ในบรรทัดที่ 11-13 เป็นการกำหนดให้ใช้ font ภาษาไทย

บรรทัดที่ 19-20 เพื่อบังคับระยะห่างระหว่างบรรทัดให้เป็น 1.5 เท่าของขนาดตัวอักษร (onehalfspacing) เพราะ singlespacing นั้นแคบไปสำหรับข้อความภาษาไทยซึ่งมีอักขระบนล่างมาก

บรรทัดที่ 22-24 คือ package ที่เทียบเท่ากับ package babel แต่รองรับ unicode และใช้กับ fontspec ได้ จุดประสงค์ในการใส่เข้ามาคือเพื่อนับเลขไทย เรียงอักษรไทย ใช้วันที่ภาษาไทยได้ โดยปกติแล้ว แนะนำให้ใช้ภาษาอังกฤษเป็นภาษาหลัก (ดังบรรทัดที่ 20) แล้วใช้ภาษาไทยเป็นภาษารองแทน  เนื่องจากแบบอักษรของตัวเลขต่างๆ จะเป็นไปตามภาษาหลัก ซึ่งโดยมากแล้วเราใช้เลขอารบิกมากกว่าเลขไทย ในเอกสารนี้จึงแสดงวันที่เป็นภาษาอังกฤษ ส่วนเลข section และลำดับ enumerate นั้นบังคับเปลี่ยนเป็นภาษาไทยใน source code เพื่อให้เห็นว่าเราสามารถนับแบบไทยก็ได้เช่นกัน

บรรทัดที่ 26 ใส่เพื่อบังคับให้ caption ของรูปภาพ ตาราง และอื่นๆ เป็นภาษาไทย เนื่องจากเมื่อตั้งภาษาอังกฤษเป็นภาษาหลัก จะทำให้ caption ต่างๆ กลายเป็นภาษาอังกฤษตามไปด้วย

ย่อหน้านี้เราจะทำการทดสอบการพิมพ์สมการ ทั้งการพิมพ์สมการในแบบ display เช่น $$(x+y)^2 = x^2 + 2xy + y^2$$  และในแบบ inline เช่น $(x+y)^2 = x^2 + 2xy + y^2$

That's all.

\end{document}
