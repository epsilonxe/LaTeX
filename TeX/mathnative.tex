%!TEX root = ../main.tex
\usepackage{mathtools,amssymb,amsthm}
% Smart Parentheses/Curly Brackets/Brackets
\newcommand{\smartp}[1]{\left( #1 \right)}
\newcommand{\smartc}[1]{\left\{ #1 \right\}}
\newcommand{\smartb}[1]{\left\langle #1 \right\rangle}

% Absolute Value
\newcommand{\abs}[1]{\left\vert #1 \right\vert}

% Vector Normal/Customized/LP Norms
\newcommand{\norm}[1]{\left\Vert #1 \right\Vert}
\newcommand{\cnorm}[2]{\left\Vert #1 \right\Vert_{#2}}
\newcommand{\lpnorm}[2]{\left\| #1 \right\|_{#2}}

% Vector Dot-Product
\newcommand{\dotproduct}[2]{\left\langle #1 , #2\right\rangle}

% Set Writing, e.g., A = {1,2,3,4,5}
\newcommand{\setwrt}[1]{\left\{ #1 \right\}}

% Set Definition, e.g., A = {x : x > 0}
\newcommand{\setdef}[2]{\left\{ #1 :\; #2 \right\}}

% Function Derivative
\newcommand{\diff}[2]{\dfrac{\mathrm{d}}{\mathrm{d}#2}#1}

% Open, Closed, Semi-Open-Closed Intervals
\newcommand{\closed}[2]{ \left[ #1,#2 \right] }
\newcommand{\open}[2]{ \left( #1,#2 \right) }
\newcommand{\closedopen}[2]{ \left[ #1,#2 \right) }
\newcommand{\openclosed}[2]{ \left( #1,#2 \right] }

% Common Math Symbols/Operators
\newcommand{\union}{\cup}
\newcommand{\intersect}{\cap}
\newcommand{\infinity}{\infty}
\newcommand{\grad}{\nabla}

% Common Math Function/Set Names
\newcommand{\sign}{\mathrm{sign}\;}
\newcommand{\id}{\mathrm{id}\;}
\newcommand{\rge}{\mathrm{rge}\;}
\newcommand{\range}{\mathrm{range}\;}
\newcommand{\interior}{\mathrm{int}\;}
\newcommand{\dom}{\mathrm{dom}\;}
\newcommand{\length}{\textrm{length}\;}
\newcommand{\prj}{\textrm{prj}\;}
