\documentclass{article}

\usepackage[thaispacing=false,thaicaption=false]{thaispec}
\usepackage{metalogo}
\usepackage{hyperref}
\hypersetup{
    colorlinks=true,
    linkcolor=black,
    filecolor=magenta,
    urlcolor=blue,
}
%\usepackage{listings}
\usepackage{color}
\usepackage{longtable}
\usepackage{minted}


\newcommand{\pkgname}{\texttt{thaispec}}
\newcommand{\showex}[1]{\par\vspace{0mm}\noindent{Example:}\par\noindent\texttt{#1}}
\newcommand{\printcenter}[1]{\par\begin{center}#1\end{center}\par\noindent}

\newcommand{\mopt}{%
frame=single,
linenos=true,
autogobble=true,
}

%\lstdefinestyle{tex}{%
%language=[LaTeX]{TeX},
%basicstyle=\ttfamily\small\color{red},
%keywordstyle=\bfseries\color{black},
%frame=single,
%backgroundcolor=\color{white},
%extendedchars=true,
%inputencoding=utf8,
%breaklines=true,
%postbreak=\mbox{\textcolor{red}{$\hookrightarrow$}\space},
%showstringspaces=true,
%}

%\lstset{style=tex}

\newminted{latex}{frame=single}

\title{The \pkgname\ package: \\Thai language typesetting in \XeLaTeX}
\author{Ratthaprom Promkam\\{\texttt{\small ratthaprom@me.com}}}
\date{Version 0.3 from March 9, 2018}

\begin{document}
\maketitle

This package allows you to input Thai characters directly to \LaTeX\ documents
and choose any (system wide) Thai fonts for typesetting in \XeLaTeX.
It also tries to appropriately justify paragraphs with no more external tools.

\tableofcontents


\section{Prerequisite}
The package use \texttt{TH Sarabun New} font by default to typeset Thai characters
which included in the collection of Thai national fonts
\footnote{Thai national fonts, a.k.a. \texttt{SIPAFonts}.
See \url{https://github.com/epsilonxe/sipafonts}}.
At least this font must be installed to system wide in order to use this package.
Moreover the following \LaTeX\ package are essentially required for the default option: \texttt{fontspec}, \texttt{uchar­classes}, \texttt{poly­glos­sia}, \texttt{setspace}, \texttt{date­time2}, \texttt{kvop­tions}, \texttt{after­pack­age}, \texttt{xstring}, and \texttt{xpatch}.

\section{Recommendation}
Install the collection of Thai national font said above and also \TeX\ Gyre font family
which possibly already included with your \TeX\ distribution.
These are basically assumed to be installed prior loading the package.

\section{Package loading}
In the preamble, add the command
\begin{minted}[frame=single]{latex}
\usepackage{thaispec}
\end{minted}
then you can input Thai characters in the document and typeset the document as usual.
By default the package set \texttt{thaifont} to \texttt{TH Sarabun New},
while set \texttt{mainfont}, \texttt{sansfont} and \texttt{monofont} to \TeX\ Gyre fonts.

In case \TeX\ Gyre font family is not system wide installed, the package should be loaded
with the following option:
\begin{minted}[frame=single]{LaTeX}
\usepackage[texgyrefont = false]{thaispec}
\end{minted}
This will typeset the document by setting \texttt{mainfont} to \texttt{TH Sarabun New}.

The package also predefines \texttt{\textbackslash today} and \texttt{\textbackslash Today}
for today Thai date printing in short and long formats respectively.

\section{Loading options}
This section lists additional loading options by their features as follows.
The examples in the list are default and also initialized values for those options.
\renewcommand{\arraystretch}{1.8}
\begin{longtable}{l p{5.8cm}}
\caption{Loading options in \texttt{thaispec} package.} \label{table:loading_options}\\
\hline
\textbf{Options}  & \textbf{Features}
\\ \hline
\endfirsthead
\caption{(continued) Loading options in \texttt{thaispec} package.}\\
\hline
\textbf{Options}  & \textbf{Features}
\\ \hline
\endhead

\hline
\endfoot
  \texttt{thainum}
  & Uses Thai numbers for almost all number digits.
  It is untoggled by defalut.
  \\
  \texttt{math}
  & Additionally load the following packages:
  \texttt{mathtools}, \texttt{amssymb}, \texttt{amsthm}, \texttt{mathspec} orderly.

  Normally \pkgname\ package loads \texttt{fontspec}\ with \texttt{no-math}\ option.
  If your document consists of math objects, this option is then recommended.
  \\
  \texttt{thaifont = <SYSTEM\_FONT\_NAME>}
  & Choose a system font for Thai characters.
  \showex{thaifont = TH Sarabun New}
  \\
  \texttt{mainfont = <SYSTEM\_FONT\_NAME>}
  & Choose a font for \texttt{mainfont} corresponding to \texttt{fontspec} package.
  \showex{thaifont = TeX Gyre Termes}
  \\
  \texttt{sansfont = <SYSTEM\_FONT\_NAME>}
  & Choose a font for \texttt{sansfont} corresponding to \texttt{fontspec} package.
  \showex{thaifont = TeX Gyre Heros}
  \\
  \texttt{monofont = <SYSTEM\_FONT\_NAME>}
  & Choose a font for \texttt{monofont} corresponding to \texttt{fontspec} package.
  \showex{thaifont = TeX Gyre Cursors}
  \\
  \texttt{thaithm = <BOOL>}
  & After loading \texttt{amsthm} package, \texttt{thaispec} package automatically defines
  a set of theorem-like environments with Thai heading by default.
  The automatic defined environments includes
  \texttt{theorem}, \texttt{lemma}, \texttt{corollary},
  \texttt{definition}, \texttt{axiom}, \texttt{undefinedterm},
  \texttt{example}, \texttt{remark} and \texttt{note}.
  If you prefer to set them yourself, just set its value to \texttt{false}.
  \showex{thaithm = true}
  \\
  \texttt{thmcount = <VALUE>}
  & If the option \texttt{thaithm = true} is prefered,
  this package set the counter independently for each automatic defined environments.
  The value of \texttt{<VALUE>} can be one of the following:
  \texttt{default}, \texttt{no}, \texttt{full}, \texttt{section},
  \texttt{chapter}, \texttt{kind}, \texttt{kind-section}, and \texttt{kind-chapter}.
  \showex{thmcount = default}
  \\
\end{longtable}

\section{Usage Examples}
The following example is a basic example of using \texttt{thaispec} package.
It is loaded with the default setting for typesetting in \XeLaTeX, i.e.,
only Thai characters are typesetted with \texttt{TH Sarabun New} font,
other charaters are typesetted with \TeX\ Gyre fonts,
and paragraphs are justified by \texttt{\textbackslash sloppy} macro.
%\begin{lstlisting}[style=tex,numbers=left]
%\documentclass{article}
%\usepackage{thaispec}
%\begin{document}
%\section{Thai ภาษาไทย}
%Thai charaters can be input directly like this ทดสอบการพิมพ์ภาษาไทยในเอกสาร \XeLaTeX\
%\end{document}
%\end{lstlisting}
\begin{minted}[
frame=single,
linenos=true,
autogobble=true,
highlightlines={2}
]{LaTeX}
\documentclass{article}
\usepackage{thaispec}
\begin{document}
\section{ภาษาไทย}
ทดสอบการพิมพ์ภาษาไทยในเอกสาร \XeLaTeX
\end{document}
\end{minted}
In order to use another Thai font face for any charaters in a math document without
\texttt{\textbackslash sloppy} macro,
the following example can be used to achieve the goal.
%\begin{lstlisting}[style=tex,numbers=left]
%\documentclass{article}
%\usepackage[math,
%thaifont = Tahoma,
%texgyrefont = false,
%sloppy = false]{thaispec}
%\begin{document}
%\section{Math ภาษาไทย}
%Thai charaters can be input directly like this ทดสอบการพิมพ์ภาษาไทยในเอกสาร $ax^2+bx+c=0$
%\end{document}
%\end{lstlisting}
\begin{minted}[%
frame=single,
linenos=true,
autogobble=true,
highlightlines={2-5}
]{LaTeX}
\documentclass{article}
\usepackage[math,
thaifont = Tahoma,
texgyrefont = false,
sloppy = false]{thaispec}
\begin{document}
\section{Math ภาษาไทย}
การพิมพ์ภาษาไทยในเอกสาร $ax^2+bx+c=0$
\end{document}
\end{minted}


\section{Known Issues}
\subsection*{Incorrect Thai characters with \texttt{listing} package}
If you typeset some codes consisting of Thai characters in \texttt{lstlisting} environment provided by \texttt{listing} package, this will possibly cause you a problem with incorrect Thai characters.
The recommendation is choosing \texttt{minted} package instead of \texttt{listing} package.
However you need to additionally install \texttt{pygments} python module in order to use \texttt{minted} package.

\section{Credits}
This package is motivated by a set of \LaTeX\ commands for typesetting Thai documents
provided by Dittaya Wanvarie
\footnote{See {\url{http://pioneer.netserv.chula.ac.th/~wdittaya/}} in \LaTeX\ section.} from Chulalongkorn University.

\section{License}
This work may be distributed and/or modified under the
conditions of the LaTeX Project Public License, either version 1.3
of this license of (at your option) any later version.
The latest version of this license is in
\printcenter{\url{http://www.latex-project.org/lppl.txt}}
and version 1.3 or later is part of all distributions of LaTeX
version 2005/12/01 or later.



\end{document}
