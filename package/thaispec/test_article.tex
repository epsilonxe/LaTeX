\documentclass{article}
\usepackage[math]{thaispec}

\title{เอกสารภาษาไทย}
\author{ชื่อผู้เขียน}



\begin{document}

\maketitle

เอกสารนี้เป็นการ
ทดสอบภาษาไทย (Thai characters) ของแพคเกจไทยสเป็คเวอร์ชัน \thaispecver เบื้องต้นบนเอกสารบทความ

\[
\int_a^b\; f(x) \;\mathrm{d}x = F(b) - F(a) \text{ เมื่อ } F^{\prime}(x) = f(x).
\]

\section{หัวเรื่อง}

เอกสารนี้เป็นการ
ทดสอบภาษาไทย (Thai characters) ของแพคเกจไทยสเป็คเวอร์ชัน \thaispecver เบื้องต้นบนเอกสารบทความ

\subsection{ทฤษฎีบทและการพิสูจน์}

\begin{definition}
สมการพหุนามกำลังสองคือสมการพหนุนามดีกรีสองที่อยู่ในรูป $ax^2+bx+c = 0$ โดยที่ $a \neq 0$, $b$ และ $c$ คือจำนวนจริง
\end{definition}

\begin{theorem}
ทดสอบภาษาไทย (Thai characters in the document) ทดสอบพิมพ์สมการ $ax^2+bx+c = 0$
\end{theorem}
\begin{proof}
    ทดสอบเอ็นวิรอนเมนท์การพิสูจน์
\end{proof}

\section{การสร้างตาราง}

\begin{table}
\centering
\begin{tabular}{llll}
\hline
หัวข้อ 1 & หัวข้อ 2  & หัวข้อ 3 & หัวข้อ 4\\
\hline
รายการ 1 & 1 & 1 & 1 \\ 
รายการ 2 & 2 & 2 & 2 \\  
รายการ 3 & 3 & 3 & 3 \\ 
\hline  
\end{tabular}
\caption{ตารางทดสอบ}    
\end{table}




\end{document}